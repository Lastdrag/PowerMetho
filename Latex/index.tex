\documentclass[12pt,a4paper]{article}
\usepackage[utf8]{inputenc}
\usepackage[T1]{fontenc}
\usepackage[spanish]{babel}
\usepackage{amsmath}
\usepackage{amsfonts}
\usepackage{amssymb}
\usepackage{graphicx}
\usepackage{listings}
\usepackage[left=1.00cm, right=1.00cm, top=1.00cm, bottom=1.00cm]{geometry}

% Default fixed font does not support bold face
\DeclareFixedFont{\ttb}{T1}{txtt}{bx}{n}{8} % for bold
\DeclareFixedFont{\ttm}{T1}{txtt}{m}{n}{8}  % for normal

% Custom colors
\usepackage{color}
\definecolor{deepblue}{rgb}{0,0,0.5}
\definecolor{deepred}{rgb}{0.6,0,0}
\definecolor{deepgreen}{rgb}{0,0.5,0}

\usepackage{listings}

% Python style for highlighting
\newcommand\pythonstyle{\lstset{
		language=Python,
		basicstyle=\ttm,
		otherkeywords={self},             % Add keywords here
		keywordstyle=\ttb\color{deepblue},
		emph={MyClass,__init__},          % Custom highlighting
		emphstyle=\ttb\color{deepred},    % Custom highlighting style
		stringstyle=\color{deepgreen},
		frame=tb,                         % Any extra options here
		showstringspaces=false,            %
		basicstyle=\scriptsize,
		numbers=left,
		stepnumber=1
}}


% Python environment
\lstnewenvironment{python}[1][]
{
	\pythonstyle
	\lstset{#1}
}
{}

% Python for external files
\newcommand\pythonexternal[2][]{{
		\pythonstyle
		\lstinputlisting[#1]{#2}}}

% Python for inline
\newcommand\pythoninline[1]{{\pythonstyle\lstinline!#1!}}

\begin{document}
	
	\title{Source code of PowerMethod}
	\maketitle

\section{Matrix.py}

\begin{python}
import numpy as np
import csv
from Util import Util

class Matrix:

def __init__(self, *args):

	# A new matrix is created by giving the args[0] parameter
	# if args[0] is an integer then it will create a square
	# matrix with this parameter.
	if isinstance(args[0],int):
		self.size = args[0]
		self.matrix = np.transpose(np.random.dirichlet(np.ones(args[0]),size=args[0]))
		self.rows, self.columns = self.matrix.shape
		
	# if args[0] is a string (the path of csv file) then it will load the
	# csv file in a new matrix
	elif isinstance(args[0],str):
		util = Util
		self.file = args[0]
		file = open(self.file)
		rows = len(file.readlines())
		file.close()
		self.rows = rows
		with open(self.file, 'r') as nFile:
			line = nFile.readline()
			self.columns = line.count(',') + 1 
			self.matrix=np.empty((self.rows,self.columns))
	
		#fill the matrix 
		with open(args[0]) as csv_file:
			csv_reader = csv.reader(csv_file, delimiter=',')
			f = 0
			for row in csv_reader:
				for c in range(self.columns):
					self.matrix[f,c] = util.stringToFloat(row[c])
				f += 1
	# creates a canonical vector
	self.canonicalVector=np.empty((self.rows,1))
	for f in range(self.rows):
		if (f == 0):
			self.canonicalVector[f,0]= 1
		else:
			self.canonicalVector[f,0]= 0   

def getNumberOfRows(self):
	return self.rows

def getNumberOfColumns(self):
	return self.columns

def getMatrix(self):
	return self.matrix

def getCanonicalVector(self):
	return self.canonicalVector;

def setCanonicalVector(self,canonicalVector):
	self.canonicalVector = canonicalVector

\end{python}

\section{Operation.py}

\begin{python}
from Matrix import Matrix
import numpy as np 

class Operation:

	def __init__(self, iterationNumber, arg):
		self.iterationNumber = iterationNumber
		self.arg = arg
		self.iteration = []
	
	def powerMethod(self):
		self.iteration = []
		self.result = None
		matrix = Matrix(self.arg)
		value=self.checkStochastic(matrix)
	
		if value == True:
			for x in range(self.iterationNumber):
				self.result=np.dot(matrix.getMatrix(),matrix.getCanonicalVector())
				matrix.setCanonicalVector(self.result)
				self.iteration.append(np.transpose(self.result))
			return True
		else:
			return False
	
	def getIteration(self):
		return self.iteration
	
	def checkStochastic(self,matrix):
		listOfValues = np.sum(matrix.getMatrix(),axis=0)
		for x in listOfValues:
			if round(x,4) != 1:
				return False
		return True
	
\end{python}

\end{document}